\chapter{The Spectral Theorem}

%   \proves{}
%   \uses{}

\begin{quotation}
  \emph{Functional Analysis by Walter Rudin 1991, extract from Chapter 12}
\end{quotation}

It should perhaps be stated explicitly that the spectrum $\sigma(T)$ of an operator $T \in \mathcal{B}(H)$ will always refer to the full algebra $\mathcal{B}(H)$.
In other words, $\lambda \in \sigma(T)$ if and only if $T - \lambda I$ has no inverse in $\mathcal{B}(H)$.
Sometimes we shall also be concerned with closed subalgebras $A$ of $\mathcal{B}(H)$ which have the additional property that $I \in A$ and $T^* \in A$ whenever $T \in A$. (Such algebras are sometimes called $*$-algebras.)

Let $A$ be such an algebra, and suppose that $T \in A$ and $T^{-1} \in \mathcal{B}(H)$.
Since $TT^*$ is self-adjoint, $\sigma(TT^*)$ is a compact subset of the real line (Theorem 12.15), hence does not separate $\mathbb{C}$, and therefore $\sigma_A(TT^*) = \sigma(TT^*)$, by the corollary to Theorem 10.18.
Since $TT^*$ is invertible in $\mathcal{B}(H)$, this equality shows that $(TT^*)^{-1} \in A$, and therefore $T^{-1} = T^(TT^*)^{-1}$ is also in $A$.

Thus $T$ has the same spectrum relative to all closed *-algebras in $\mathcal{B}(H)$ that contain $T$.

Theorem 12.23 will be obtained as a special case of the following result, which deals with normal algebras of operators rather than with individual ones.

\begin{theorem}[12.22]

  If $A$ is a closed normal subalgebra of $\mathcal{B}(H)$ which contains the identity operator $I$ and if $\Delta$ is the maximal ideal space of $A$, then the following assertions are true:

  \begin{enumerate}
    \item \label{itm:a} There exists a unique resolution $E$ of the identity on the Borel subsets of $\Delta$ which satisfies
          \begin{equation}
            \label{eq:1}
            T = \int_\Delta \widehat{T} \ dE
          \end{equation}
          for every $T \in A$, where $\widehat{T}$ is the Gelfand transform of $T$.
    \item \label{itm:b} The inverse of the Gelfand transform (i.e., the map that takes $\widehat{T}$ back to $T$) extends to an isometric *-isomorphism of the algebra \(L^\infty(E)\) onto a closed subalgebra $B$ of $\mathcal{B}(H)$, $B\supset A$, given by
          \begin{equation}
            \label{eq:2}
            \Phi f = \int_\Delta f \ dE \quad (f \in L^\infty(E)).
          \end{equation}
          Explicitly, $\Phi$ is linear and multiplicative and satisfies
          \[
            \Phi(\bar{f}) = (\Phi f)^*, \| \Phi f \| = \| f \|_{\infty} \quad (f \in L^\infty(E)).
          \]
    \item \label{itm:c}  $B$ is the closure [in the norm topology of  $\mathcal{B}(H)$] of the set of all finite linear combinations of the projections $E(\omega)$.
    \item \label{itm:d}  If $\omega \subset \Delta$ is open and nonempty, then $E(\omega) \neq 0$.
    \item \label{itm:e}  An operator $S \in \mathcal{B}(H)$ commutes with every $T \in A$ if and only if $S$ com mutes with every projection $E(\omega)$.
  \end{enumerate}
\end{theorem}

\begin{proof}
  Recall that \eqref{eq:1} is an abbreviation for
  \begin{equation}
    \label{eq:4}
    (Tx , y) = \int_\Delta \widehat{T} \ dE_{x,y} \quad (x,y \in H, T \in A).
  \end{equation}
  Since $\mathcal{B}(H)$ is a $B^*$-algebra (Section 12.9), our given algebra $A$ is a commutative $B^*$-algebra.
  The Gelfand-Naimark theorem 11.18 asserts therefore that $T \to \widehat{T}$ is an isometric *-isomorphism of $A$ onto $C(\Delta)$.

  This leads to an easy proof of the uniqueness of $E$.
  Suppose $E$ satisfies \eqref{eq:4}.
  Since $\widehat{T}$ ranges over all of $C(\Delta)$, the assumed regularity of the complex Borel measures $E_{x,y}$ shows that each $E_{x,y}$ is uniquely determined by \eqref{eq:4}; this follows from the uniqueness assertion that is part of the Riesz representation theorem ([23], Th. 6.19).
  Since, by definition, $(E(\omega)x, y) = E_{x,y}(\omega)$, each projection $E(\omega))$ is also uniquely determined by \eqref{eq:4}.

  This uniqueness proof motivates the following proof of the existence of $E$.
  If $x \in H$ and $y \in H$, Theorem 11.18 shows that $ \widehat{T} \mapsto (Tx, y)$ is a bounded linear functional on $C(\Delta)$, of norm $\leq \|x\|| \|y\|$, since $\| \widehat{T}\|_{\infty} = \|T\|$.
  The Riesz representation theorem supplies us therefore with unique regular complex Borel measures $\mu_{x,y}$ on $\Delta$ such that
  \begin{equation}
    \label{eq:5}
    (Tx , y) = \int_\Delta \widehat{T} \ d\mu_{x,y} \quad (x,y \in H, T \in A).
  \end{equation}
  For fixed \( T \), the left side of \eqref{eq:5} is a bounded sesquilinear functional on \( H \), hence so is the right side, and it remains so if the continuous function
  \( \widehat{T} \) is replaced by an arbitrary bounded Borel function \( f \).
  To each such \( f \) corresponds therefore an operator \( \Phi f \in \mathcal{B}(H) \) (see Theorem 12.8) such that

  \begin{equation}
    \label{eq:6}
    ((\Phi f)x, y) = \int_{\Delta} f \ d\mu_{x,y} \quad (x, y \in H).
  \end{equation}

  Comparison of \eqref{eq:5} and \eqref{eq:6} shows that \( \Phi \hat{T} = T \). Thus \( \Phi \) is an extension of the inverse of the Gelfand transform.

  It is clear that \( \Phi \) is linear.

  Part of the Gelfand-Naimark theorem states that \( T \) is self-adjoint if and only if \( \hat{T} \) is real-valued. For such \( T \),

  \[
    \int_{\Delta} \widehat{T} \ d\mu_{x,y} = (Tx, y) = (x, Ty) = \overline{(Ty, x)} = \overline{\int_{\Delta} \hat{T} d\mu_{y,x}},
  \]

  and this implies that \( \mu_{y,x} = \overline{\mu_{x,y}} \).
  Hence,

  \[
    ((\Phi \overline{f})x, y) = \int_{\Delta} \bar{f} \ d\mu_{x,y} = \overline{\int_{\Delta} f \, d\mu_{y,x}} = \overline{((\Phi f)y, x)} = (x, (\Phi f)y)
  \]

  for all \( x, y \in H \), so that

  \begin{equation}
    \label{eq:7}
    \Phi \bar{f} = (\Phi f)^*.
  \end{equation}

  Our next objective is the equality

  \begin{equation}
    \label{eq:8}
    \Phi (fg) = (\Phi f)(\Phi g)
  \end{equation}

  for bounded Borel functions \( f, g \) on \( \Delta \). If \( S \in A \) and \( T \in A \), then
  \( (ST)^{\wedge} = \widehat{S} \widehat{T} \);
  hence

  \[
    \int_{\Delta} \hat{S} \hat{T} \ d\mu_{x,y} = (STx, y) = \int_{\Delta} \widehat{S} \ d\mu_{Tx,y}.
  \]

  This holds for every \( \widehat{S} \in C(\Delta) \);
  hence the two integrals are equal if \( \widehat{S} \) is
  replaced by any bounded Borel function \( f \). Thus

  \[
    \int_{\Delta} f \widehat{T} d\mu_{x,y} = \int_{\Delta} f \ d\mu_{Tx,y} = ((\Phi f)Tx, y) = (Tx, z) = \int_{\Delta} \hat{T} d\mu_{x,z},
  \]

  where we put \( z = (\Phi f)^* y \).
  Again, the first and last integrals remain
  equal if \( \widehat{T} \) is replaced by \( g \).
  This gives

  \[
    \begin{aligned}
      (\Phi (fg)x, y) & = \int_{\Delta} fg \ d\mu_{x,y} = \int_{\Delta} g \ d\mu_{x,z}          \\
                      & = ((\Phi g)x, z) = ((\Phi g)x, (\Phi f)^* y) = (\Phi (f) \Phi (g)x, y),
    \end{aligned}
  \]

  and \eqref{eq:8} is proved.

  We are finally ready to define \( E \): If \( \omega \) is a Borel subset of \( \Delta \), let
  \( \chi_{\omega} \) be its characteristic function, and put

  \[
    E(\omega) = \Phi(\chi_{\omega}).
  \]

  By \eqref{eq:8}, \( E(\omega \cap \omega') = E(\omega)E(\omega') \).
  With \( \omega' = \omega \), this shows that
  each \( E(\omega) \) is a projection.
  Since \( \Phi f \) is self-adjoint when \( f \) is real, by \eqref{eq:7},
  each \( E(\omega) \) is self-adjoint.
  It is clear that \( E(\emptyset) = \Phi(0) = 0 \).
  That
  \( E(\Delta) = I \) follows from \eqref{eq:5} and \eqref{eq:6}.
  The finite additivity of \( E \) is a consequence
  of \eqref{eq:6}, and, for all \( x, y \in H \),

  \[
    E_{x,y}(\omega) = (E(\omega)x, y) = \int_{\Delta} \chi_{\omega} \ d\mu_{x,y} = \mu_{x,y}(\omega).
  \]

  Thus \eqref{eq:6} becomes \eqref{eq:2}.
  That \( \|\Phi f\| = \|f\|_{\infty} \) follows now from Theorem 12.21.

  This completes the proof of (\ref{itm:a}) and (\ref{itm:b}).

  Part (\ref{itm:c}) is now clear because every \( f \in L^{\infty}(E) \) is a uniform limit of
  simple functions (i.e., of functions with only finitely many values).

  Suppose next that \( \omega \) is open and \( E(\omega) = 0 \).
  If \( T \in A \) and \( \widehat{T} \) has
  its support in \( \omega \), \eqref{eq:1} implies that \( T = 0 \); hence \( \widehat{T} = 0 \).
  Since \( \widehat{A} = C(\Delta) \),
  Urysohn's lemma implies now that \( \omega = \emptyset \).
  This proves (\ref{itm:d}).

  To prove (\ref{itm:e}), choose \( S \in \mathcal{B}(H) \), \( x \in H \), \( y \in H \), and put \( z = S^* y \).
  For any \( T \in A \) and any Borel set \( \omega \subset \Delta \) we then have

  \begin{equation}
    \label{eq:10}
    (STx, y) = (Tx, z) = \int_{\Delta} \widehat{T} \ dE_{x,z},
  \end{equation}

  \begin{equation}
    \label{eq:11}
    (TSx, y) = \int_{\Delta} \hat{T} \ dE_{Sx,y},
  \end{equation}

  \[
    (SE(\omega)x, y) = (E(\omega)x, z) = E_{x,z}(\omega),
  \]

  \[
    (E(\omega)Sx, y) = E_{Sx,y}(\omega).
  \]

  If \( ST = TS \) for every \( T \in A \), the measures in \eqref{eq:10} and \eqref{eq:11} are
  equal, so that \( SE(\omega) = E(\omega)S \).
  The same argument establishes the converse.
  This completes the proof.
\end{proof}
