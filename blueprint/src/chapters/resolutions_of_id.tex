\chapter{Resolutions of the identity}

\begin{definition}[12.17]
  Let $\mathfrak{M}$ be a $\sigma$-algebra in a set $\Omega$, and let $H$ be a Hilbert space.
  For simplicity, we assume that $\Omega$ is a locally compact (Hausdorff) space.
  In this setting, a \emph{resolution of the identity} (on $\mathfrak{M}$) is a mapping

  \[
    E: \mathfrak{M} \to \mathfrak{M}(H)
  \]
  with the following properties:

  \begin{enumerate}
    \item \label{itm:Ra} \( E(\emptyset) = 0\), \(E(\Omega) = I\).
    \item \label{itm:Rb}  Each \( E(\omega) \) is a self-adjoint projection.
    \item \label{itm:Rc} \( E(\omega' \cap \omega'') = E(\omega')E(\omega'')\).
    \item \label{itm:Rd}  If \( \omega' \cap \omega'' = \emptyset \), then \( E(\omega' \cup \omega'') = E(\omega') + E(\omega'') \).
    \item \label{itm:Re}  For every \( x \in H \) and \( y \in H \), the set function \( E_{x,y} \) defined by:
          \[
            E_{x,y}(\omega) = (E(\omega)x, y)
          \]
          is a complex regular Borel measure on \( \mathcal{M} \).
  \end{enumerate}
\end{definition}

% When \( \mathfrak{M} \) is the \( \sigma \)-algebra of all Borel sets on a compact or locally compact Hausdorff space, it is customary to add another requirement to (\ref{itm:Re}):
% Each \( E_{x,y} \) should be a regular Borel measure.
% (This is automatically satisfied on compact metric spaces, for instance. See [23].)

\begin{lemma}
 For any $x \in H$,
 \[
  E_{x,x}(\omega) = (E(\omega)x, x) = \|E(\omega)x\|^2.
 \]
\end{lemma}

\begin{lemma}
 For any $x \in H$,
 \( E_{x,x} \) is a positive measure on \( \mathfrak{M} \) whose total variation is:
 \[
  \|E_{x,x}\| = E_{x,x}(\Omega) = \|x\|^2.
 \]
\end{lemma}

\begin{lemma}
 For two $\omega_1, \omega_2$, \( E(\omega_1), E(\omega_2) \) commute.
\end{lemma}
\begin{proof}
 By (\ref{itm:Rc}), any two of the projections \( E(\omega) \) commute with each other.
\end{proof}

\begin{lemma}
 If \( \omega' \cap \omega'' = \emptyset \),
 then the ranges of \( E(\omega') \) and \( E(\omega'') \) are orthogonal to each other
\end{lemma}
\begin{proof}
 By (\ref{itm:Ra}), (\ref{itm:Rc}) and Theorem 12.14.
\end{proof}

\begin{lemma}
 If $\{\omega_j\}$ is a finite family of mutually disjoint Borel sets, then
 $E(\bigcup_j \omega_j) = \sum_j E(\omega_j)$.
\end{lemma}
\begin{proof}
 By (\ref{itm:Rd}) and induction.
\end{proof}

\textbf{Remark:}
$\sum_{n=1}^{\infty} E(\omega_n)$ does not converge in the norm topology of $\mathcal{B}(H)$.

\begin{lemma}
 Let $x \in H$ and $\{\omega_j\}$ be a countable family of mutually disjoint Borel sets.
 Then $E(\bigcup_j \omega_j)x = \sum_j E(\omega_j)x$, where
 the right-hand side converges in the norm topology of $H$.
\end{lemma}
\begin{proof}
 Since \( E(\omega_n)E(\omega_m) = 0 \) when \( n \neq m \), the vectors \( E(\omega_n)x \) and \( E(\omega_m)x \)
 are orthogonal to each other (Theorem 12.14).
 By (\ref{itm:Re}),
 \begin{equation}
   \label{eq:R5}
   \sum_{n=1}^{\infty} (E(\omega_n)x, y) = (E(\omega)x, y)
 \end{equation}
 for every \( y \in H \).
 It now follows from Theorem 12.6 that:
 \[
  \sum_{n=1}^{\infty} E(\omega_n)x = E(\omega)x.
 \]
 The series (\eqref{eq:R5}) converges in the norm topology of \( H \).
\end{proof}

\begin{proposition}[12.18]
  If \( E \) is a resolution of the identity, and if \( x \in H \), then
  \[
    \omega \mapsto E(\omega)x
  \]
  is a countably additive \( H \)-valued measure on* \( \mathfrak{M} \).
\end{proposition}

Moreover, sets of measure zero can be handled in the usual way:

\begin{proposition}[12.19]
  Suppose \( E \) is a resolution of the identity.
  If \( \omega_n \in \mathfrak{M} \) and \( E(\omega_n) = 0 \) for \( n = 1,2,3,\dots \), and if
  \[
    \omega = \bigcup_{n=1}^{\infty} \omega_n,
  \]
  then \( E(\omega) = 0 \).
\end{proposition}

\begin{proof}
  Since \( E(\omega_n) = 0 \), \( E_{x,x}(\omega_n) = 0 \) for every \( x \in H \). Since \( E_{x,x} \) is countably additive, it follows that \( E_{x,x}(\omega) = 0 \). But
  \[
    \|E(\omega)x\|^2 = E_{x,x}(\omega).
  \]
  Hence, \( E(\omega) = 0 \).
\end{proof}


% **12.20 The algebra \( L^{\infty}(E) \)**
% Let \( E \) be a resolution of the identity on \( \mathcal{M} \), as above. Let \( f \) be a complex \( \mathcal{M} \)-measurable function on \( \Omega \). There is a countable collection \( \{D_i\} \) of open discs that forms a base for the topology of \( \mathbb{C} \). Let \( V \) be the union of those \( D_i \) for which \( E(f^{-1}(D_i)) = 0 \). By Proposition 12.19, \( E(f^{-1}(V)) = 0 \). Also, \( V \) is the largest open subset of \( \mathbb{C} \) with this property.

% The **essential range** of \( f \) is, by definition, the complement of \( V \). It is the smallest closed subset of \( \mathbb{C} \) that contains \( f(p) \) for almost all \( p \in \Omega \), that is, for all \( p \in \Omega \) except those that lie in some set \( \omega \in \mathcal{M} \) with \( E(\omega) = 0 \).

% We say that \( f \) is **essentially bounded** if its essential range is bounded, hence compact. In that case, the largest value of \( |\lambda| \), as \( \lambda \) runs through the essential range of \( f \), is called the **essential supremum** \( \| f \|_{\infty} \) of \( f \).

% Let \( B \) be the algebra of all bounded complex \( \mathcal{M} \)-measurable functions on \( \Omega \), with the norm:

% \[
%   \| f \| = \sup \{ | f(p) | : p \in \Omega \}.
% \]

% One sees easily that \( B \) is a Banach algebra and that:

% \[
%   N = \{ f \in B : \| f \|_{\infty} = 0 \}
% \]

% is an ideal of \( B \) which is **closed**, by Proposition 12.19. Hence, \( B/N \) is a Banach algebra, which we denote (in the usual manner) by \( L^{\infty}(E) \).

% The norm of any coset \( [f] = f + N \) of \( L^{\infty}(E) \) is then equal to \( \| f \|_{\infty} \), and its spectrum \( \sigma([f]) \) is the essential range of \( f \). As is usually done in measure theory, the distinction between \( f \) and its equivalence class \( [f] \) will be ignored.

% Our next concern will be the integration of functions with respect to the projection-valued measures described above. The resulting integrals
% \(\int f \, dE\) turn out to be not only linear (as all good integrals ought to be) but also multiplicative!

% **12.21 Theorem**
% *If \( E \) is a resolution of the identity, as above, then there exists an isometric*-isomorphism* \( \Psi \) *of the Banach algebra* \( L^{\infty}(E) \) *onto a closed normal subalgebra* \( A \) *of* \( \mathcal{B}(H) \), *which is related to* \( E \) *by the formula.*

% \[
% (\Psi(f)x, y) = \int_{\Omega} f \, dE_{x,y} \quad \text{for } (x, y \in H, f \in L^{\infty}(E)).
% \]

% This justifies the notation

% \[
% \Psi(f) = \int_{\Omega} f \, dE.
% \]

% Moreover,

% \[
% \|\Psi(f)x\|^2 = \int_{\Omega} |f|^2 \, dE_{x,x} \quad \text{for } (x \in H, f \in L^{\infty}(E)).
% \]

% *And an operator* \( Q \in \mathcal{B}(H) \) *commutes with every* \( E(\omega) \) *if and only if* \( Q \) *commutes with every* \( \Psi(f) \).

% ---

% Recall that a **normal subalgebra** \( A \) of \( \mathcal{B}(H) \) is a commutative one which contains \( T^* \) for every \( T \in A \). To say that \( \Psi \) is a \( * \)-isomorphism means that \( \Psi \) is one-to-one, linear, and multiplicative and that

% \[
% \Psi(\bar{f}) = \Psi(f)^* \quad \text{for } (f \in L^{\infty}(E)).
% \]

% ---

% **Proof**
% To begin with, let \( \{\omega_1, \dots, \omega_n\} \) be a partition of \( \Omega \), with \( \omega_i \in \mathcal{M} \), and let \( s \) be a simple function, such that \( s = \alpha_i \) on \( \omega_i \). Define \( \Psi(s) \in \mathcal{B}(H) \) by

% \[
% \Psi(s) = \sum_{i=1}^{n} \alpha_i E(\omega_i).
% \]

% Since each \( E(\omega_i) \) is self-adjoint,

% \[
% \Psi(s)^* = \sum_{i=1}^{n} \bar{\alpha}_i E(\omega_i) = \Psi(\bar{s}).
% \]

% If \( \{\omega'_1, \dots, \omega'_m\} \) is another partition of this kind, and if \( t = \beta_j \) on \( \omega'_j \), then

% \[
% \Psi(s) \Psi(t) = \sum_{i,j} \alpha_i \beta_j E(\omega_i) E(\omega'_j) = \sum_{i,j} \alpha_i \beta_j E(\omega_i \cap \omega'_j).
% \]

% Since \( st \) is the simple function that equals \( \alpha_i \beta_j \) on \( \omega_i \cap \omega_j \), it follows that

% \[
% \Psi(s) \Psi(t) = \Psi(st).
% \]

% An entirely analogous argument shows that

% \[
% \Psi(\alpha s + \beta t) = \alpha \Psi(s) + \beta \Psi(t).
% \]

% If \( x \in H \) and \( y \in H \), (5) leads to

% \[
% (\Psi(s)x, y) = \sum_{i=1}^{n} \alpha_i (E(\omega_i)x, y) = \sum_{i=1}^{n} \alpha_i E_{x,y}(\omega_i) = \int_{\Omega} s \, dE_{x,y}.
% \]

% By (6) and (7),

% \[
% \Psi(s)^* \Psi(s) = \Psi(\bar{s}) \Psi(s) = \Psi(\bar{s} s) = \Psi(|s|^2).
% \]

% Hence (9) yields

% \[
% \|\Psi(s)x\|^2 = (\Psi(s)^* \Psi(s)x, x) = (\Psi(|s|^2)x, x) = \int_{\Omega} |s|^2 dE_{x,x},
% \]

% so that

% \[
% \|\Psi(s)x\| \leq \|s\|_{\infty} \|x\|,
% \]

% by formula (2) of Section 12.17. On the other hand, if \( x \in \mathcal{R}(E(\omega_j)) \), then

% \[
% \Psi(s)x = \alpha_j E(\omega_j)x = \alpha_j x,
% \]

% since the projections \( E(\omega_i) \) have mutually orthogonal ranges. If \( j \) is chosen so that \( |\alpha_j| = \|s\|_{\infty} \), it follows from (12) and (13) that

% \[
% \|\Psi(s)\| = \|s\|_{\infty}.
% \]

% Now suppose \( f \in L^{\infty}(E) \). There is a sequence of simple measurable functions \( s_k \) that converges to \( f \) in the norm of \( L^{\infty}(E) \). By (14), the corresponding operators \( \Psi(s_k) \) form a Cauchy sequence in \( \mathcal{B}(H) \) which is therefore norm-convergent to an operator that we call \( \Psi(f) \); it is easy to see that \( \Psi(f) \) does not depend on the particular choice of \( \{s_k\} \). Obviously (14) leads to

% \[
% \|\Psi(f)\| = \|f\|_{\infty}, \quad (f \in L^{\infty}(E)).
% \]

% Now (1) follows from (9) (with \( s_k \) in place of \( s \)), since each \( E_{x,x} \) is a finite measure; (2) and (3) follow from (6) and (11); and if bounded measurable functions \( f \) and \( g \) are approximated, in the norm of \( L^{\infty}(E) \), by simple measurable functions \( s \) and \( t \), we see that (7) and (8) hold with \( f \) and \( g \) in place of \( s \) and \( t \).

% Thus Ψ is an isometric isomorphism of \( L^\infty(E) \) into \( \mathcal{B}(H) \). Since \( L^\infty(E) \) is complete, its image \( A = \Psi(L^\infty(E)) \) is closed in \( \mathcal{B}(H) \), because of (15).

% Finally, if \( Q \) commutes with every \( E(\omega) \), then \( Q \) commutes with \( \Psi(s) \) whenever \( s \) is simple, and therefore the approximation process used above shows that \( Q \) commutes with every member of \( A \). \(\quad \////\)

% It is perhaps worth mentioning that the equality

% \[
% (16) \quad \| f \|_\infty^2 = \sup \left\{ \int_\Omega |f|^2 \, dE_{x,x} : \| x \| \leq 1 \right\}
% \]

% holds for every \( f \in L^\infty(E) \), because of (3) and (15).
