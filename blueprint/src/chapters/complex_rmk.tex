\chapter{The Reisz Theorem}

\begin{itemize}
  \item Destination: MeasureTheory/Integral/RieszMarkovKakutam/ComplexRMK
  \item Principal reference: Theorem 6.19 of [Walter Rudin, Real and Complex Analysis.][Rud87].
\end{itemize}

The main statement is:

\noindent
\fbox{\parbox{\linewidth}{
    If $X$ is a locally compact Hausdorff space, then every bounded linear functional $\Phi$ on $C_0(X)$ is represented by a unique regular complex Borel measure $\mu$, in the sense that
    \begin{equation*}
      \Phi f = \int_X f \, d\mu
    \end{equation*}
    for every $f \in C_0(X)$.
    Moreover, the norm of $\Phi$ is the total variation of $\mu$:
    \begin{equation*}
      \|\Phi\| = |\mu|(X).
    \end{equation*}
  }}

\begin{definition}[Variation of a Vector-Valued Measure]
  \label{def:variation}
  \lean{MeasureTheory.VectorMeasure.variation}
  \leanok
  Let $(X, \mathcal{A})$ be a measurable space and let $Y$ be a Banach space. For a vector-valued measure $\mu: \mathcal{A} \to Y$, the \textbf{variation} of $\mu$ is the set function $|\mu|: \mathcal{A} \to [0, +\infty]$ defined by
  \begin{equation*}
    |\mu|(E) = \sup \left\{ \sum_{i=1}^n \|\mu(E_i)\|_Y : \{E_1, E_2, \ldots, E_n\} \text{ is a finite partition of } E \text{ in } \mathcal{A} \right\}
  \end{equation*}
  for each $E \in \mathcal{A}$.
\end{definition}

Equivalently, the above definition can be written as:
\begin{equation*}
  |\mu|(E) = \sup \left\{ \sum_{i=1}^n \|\mu(E_i)\|_Y : E_i \in \mathcal{A}, \, E_i \cap E_j = \emptyset \text{ for } i \neq j, \, \bigcup_{i=1}^n E_i \subseteq E \right\}
\end{equation*}

\begin{theorem}[Rudin 6.12 (polar representation of a complex measure)]
  \label{thm:polar_rep}
  Let $\mu$ be a complex measure on a $\sigma$-algebra $\mathfrak{M}$ in $X$. Then there is a measurable function $h$ such that $|h(x)| = 1$ for all $x \in X$ and such that
  \begin{equation}
    d\mu = h \, d|\mu|.
  \end{equation}
\end{theorem}

\begin{proof}
  \uses{def:variation}
  This rather depends on how the integral with respect to a complex measure is defined.
  See [Rudin, Theorem 6.12] for details.
\end{proof}


\begin{definition}
  \label{def:riesz_measure}
  \lean{ComplexRMK.rieszMeasure}
  \notready
  Let $X$ be a locally compact Hausdorff space.
  Associated to every bounded linear functional $\Phi$ on $C_0(X)$ we define a regular complex Borel measure $\mu$ which we call the Riesz Measure associated to $\Phi$.

  TO DO: insert details from the proof of the exact definition.
\end{definition}

In order to prove the main result we divide the result into several smaller results.

\begin{lemma}
  \label{lem:rieszMeasure_unique}
  \lean{ComplexRMK.rieszMeasure_unique}
  Let $X$ be a locally compact Hausdorff space, and let $\Phi$ be a bounded linear functional on $C_0(X)$.
  Suppose that $\mu$, $\nu$ are regular complex Borel measure such that
  \begin{equation*}
    \Phi f = \int_X f \, d\mu = \int_X f \, d\nu.
  \end{equation*}
  Then \(\mu = \nu\).
\end{lemma}

\begin{theorem}[Rudin 3.14]
  \label{thm:Cc_dense_Lp}
  For $1 \leq p < \infty$, $C_c(X)$ is dense in $L^p(\mu)$.
\end{theorem}

\begin{proof}
  Define $S$ as in Theorem 3.13. If $s \in S$ and $\varepsilon > 0$, there exists a $g \in C_c(X)$ such that $g(x) = s(x)$ except on a set of measure $< \varepsilon$, and $|g| \leq \|s\|_\infty$ (Lusin's theorem).
  Hence
  \begin{equation}
    \|g - s\|_p \leq 2\varepsilon^{1/p}\|s\|_\infty.
  \end{equation}
  Since $S$ is dense in $L^p(\mu)$, this completes the proof.
\end{proof}

\begin{theorem}[Rudin 6.13]
  \label{thm:abs_integral}
  Suppose $\mu$ is a positive measure on $\mathfrak{M}$, $g \in L^1(\mu)$, and
  \begin{equation}
    \lambda(E) = \int_E g \, d\mu \quad (E \in \mathfrak{M}).
  \end{equation}
  Then
  \begin{equation}
    |\lambda|(E) = \int_E |g| \, d\mu \quad (E \in \mathfrak{M}).
  \end{equation}
\end{theorem}

\begin{proof}
  \uses{thm:polar_rep}
  By Theorem \ref{thm:polar_rep}, there is a function $h$, of absolute value 1, such that $d\lambda = h \, d|\lambda|$. By hypothesis, $d\lambda = g \, d\mu$. Hence
  \[
    h \, d|\lambda| = g \, d\mu.
  \]
  This gives $d|\lambda| = \bar{h}g \, d\mu$. (Compare with Theorem 1.29.)
  Since $|\lambda| \geq 0$ and $\mu \geq 0$, it follows that $\bar{h}g \geq 0$ a.e. $[\mu]$, so that $\bar{h}g = |g|$ a.e. $[\mu]$. \qed
\end{proof}

\begin{theorem}[Rudin 6.16]
  \label{thm:Lp_duality}
  Suppose $1 \leq p < \infty$, $\mu$ is a $\sigma$-finite positive measure on $X$, and $\Phi$ is a bounded linear functional on $L^p(\mu)$. Then there is a unique $g \in L^q(\mu)$, where $q$ is the exponent conjugate to $p$, such that
  \begin{equation}
    \Phi(f) = \int_X fg \, d\mu \quad (f \in L^p(\mu)).
  \end{equation}
  Moreover, if $\Phi$ and $g$ are related as in (1), we have
  \begin{equation}
    \|\Phi\| = \|g\|_q.
  \end{equation}

  In other words, $L^q(\mu)$ is isometrically isomorphic to the dual space of $L^p(\mu)$, under the stated conditions.
\end{theorem}
6.16: Duality of $L^1$ and $L^∞$ (not in Mathlib
\url{https://leanprover.zulipchat.com/#narrow/channel/217875-Is-there-code-for-X.3F/topic/Lp.20duality/near/495207025}

\begin{proof}
  \uses{def:riesz_measure}
  \uses{thm:polar_rep}
  \uses{thm:Cc_dense_Lp}
  Suppose $\mu$ is a regular complex Borel measure on $X$ and $\int f \, d\mu = 0$ for all $f \in C_0(X)$.
  By Theorem \ref{thm:polar_rep} there is a Borel function $h$, with $|h| = 1$, such that $d\mu = h \, d|\mu|$.
  For any sequence $\{f_n\}$ in $C_0(X)$ we then have
  \begin{equation}
    |\mu|(X) = \int_X (\bar{h} - f_n)h \, d|\mu| \leq \int_X |\bar{h} - f_n| \, d|\mu|, \tag{3}
  \end{equation}
  and since $C_c(X)$ is dense in $L^1(|\mu|)$ (Theorem \ref{thm:Cc_dense_Lp}), $\{f_n\}$ can be so chosen that the last expression in (3) tends to 0 as $n \to \infty$.
  Thus $|\mu|(X) = 0$, and $\mu = 0$.
  It is easy to see that the difference of two regular complex Borel measures on $X$ is regular.
  This shows that at most one $\mu$ corresponds to each $\Phi$.
\end{proof}


\begin{lemma}
  \label{lem:exists_pos_lin_func}
  \lean{ComplexRMK.exists_pos_lin_func}
  Consider a given bounded linear functional $\Phi$ on $C_0(X)$.
  Assume $\|\Phi\| = 1$. (Update statement to be the general case.)
  We shall construct a positive linear functional $\Lambda$ on $C_c(X)$, such that
  \begin{equation}
    |\Phi(f)| \leq \Lambda(|f|) \leq \|f\| \quad (f \in C_c(X)), \tag{4}
  \end{equation}
  where $\|f\|$ denotes the supremum norm.
\end{lemma}

\begin{proof}
  Assume $\|\Phi\| = 1$, without loss of generality.

  So all depends on finding a positive linear functional $\Lambda$ that satisfies (4). If $f \in C_c^+(X)$ [the class of all nonnegative real members of $C_c(X)$], define
  \begin{equation}
    \Lambda f = \sup \{|\Phi(h)| : h \in C_c(X), |h| \leq f\}. \tag{9}
  \end{equation}

  Then $\Lambda f \geq 0$, $\Lambda$ satisfies (4), $0 \leq f_1 \leq f_2$ implies $\Lambda f_1 \leq \Lambda f_2$, and $\Lambda(cf) = c\Lambda f$ if $c$ is a positive constant. We have to show that
  \begin{equation}
    \Lambda(f + g) = \Lambda f + \Lambda g \quad (f \text{ and } g \in C_c^+(X)), \tag{10}
  \end{equation}
  and we then have to extend $\Lambda$ to a linear functional on $C_c(X)$.

  Fix $f$ and $g \in C_c^+(X)$. If $\varepsilon > 0$, there exist $h_1$ and $h_2 \in C_c(X)$ such that $|h_1| \leq f$, $|h_2| \leq g$, and
  \begin{equation}
    \Lambda f \leq |\Phi(h_1)| + \varepsilon, \quad \Lambda g \leq |\Phi(h_2)| + \varepsilon. \tag{11}
  \end{equation}

  There are complex numbers $\alpha_i$, $|\alpha_i| = 1$, so that $\alpha_i \Phi(h_i) = |\Phi(h_i)|$, $i = 1, 2$. Then
  \begin{align}
    \Lambda f + \Lambda g & \leq |\Phi(h_1)| + |\Phi(h_2)| + 2\varepsilon      \\
                          & = \Phi(\alpha_1 h_1 + \alpha_2 h_2) + 2\varepsilon \\
                          & \leq \Lambda(|h_1| + |h_2|) + 2\varepsilon         \\
                          & \leq \Lambda(f + g) + 2\varepsilon,
  \end{align}
  so that the inequality $\geq$ holds in (10).

  Next, choose $h \in C_c(X)$, subject only to the condition $|h| \leq f + g$, let $V = \{x : f(x) + g(x) > 0\}$, and define
  \begin{align}
    h_1(x) & = \frac{f(x)h(x)}{f(x) + g(x)}, \quad h_2(x) = \frac{g(x)h(x)}{f(x) + g(x)} \quad (x \in V), \tag{12} \\
    h_1(x) & = h_2(x) = 0 \quad (x \notin V).
  \end{align}

  It is clear that $h_1$ is continuous at every point of $V$.
  If $x_0 \notin V$, then $h(x_0) = 0$; since $h$ is continuous and since $|h_1(x)| \leq |h(x)|$ for all $x \in X$, it follows that $x_0$ is a point of continuity of $h_1$.
  Thus $h_1 \in C_c(X)$, and the same holds for $h_2$.

  Since $h_1 + h_2 = h$ and $|h_1| \leq f$, $|h_2| \leq g$, we have
  \begin{equation}
    |\Phi(h)| = |\Phi(h_1) + \Phi(h_2)| \leq |\Phi(h_1)| + |\Phi(h_2)| \leq \Lambda f + \Lambda g.
  \end{equation}

  Hence $\Lambda(f + g) \leq \Lambda f + \Lambda g$, and we have proved (10).

  If $f$ is now a real function, $f \in C_c(X)$, then $2f^+ = |f| + f$, so that $f^+ \in C_c^+(X)$; likewise, $f^- \in C_c^+(X)$; and since $f = f^+ - f^-$, it is natural to define
  \begin{equation}
    \Lambda f = \Lambda f^+ - \Lambda f^- \quad (f \in C_c(X), f \text{ real}) \tag{13}
  \end{equation}
  and
  \begin{equation}
    \Lambda(u + iv) = \Lambda u + i\Lambda v. \tag{14}
  \end{equation}

  Simple algebraic manipulations, just like those which occur in the proof of Theorem 1.32, show now that our extended functional $\Lambda$ is linear on $C_c(X)$.
\end{proof}

\begin{theorem}[Rudin 6.19]
  \label{integral_rieszMeasure}
  \lean{ComplexRMK.integral_rieszMeasure}
  If $X$ is a locally compact Hausdorff space, then every bounded linear functional $\Phi$ on $C_0(X)$ is represented by a regular complex Borel measure $\mu$, in the sense that
  \begin{equation}
    \Phi f = \int_X f \, d\mu \tag{1}
  \end{equation}
  for every $f \in C_0(X)$.
\end{theorem}

\begin{proof}
  \uses{lem:exists_pos_lin_func}
  \uses{thm:Lp_duality}
  Once we have the $\Lambda$ from Lemma~\ref{lem:exists_pos_lin_func}, we associate with it a positive Borel measure $\lambda$, as in Theorem 2.14.
  The conclusion of Theorem 2.14 shows that $\lambda$ is regular if $\lambda(X) < \infty$.
  Since
  \begin{equation}
    \lambda(X) = \sup \{\Lambda f : 0 \leq f \leq 1, f \in C_c(X)\}
  \end{equation}
  and since $|\Lambda f| \leq 1$ if $\|f\| \leq 1$, we see that actually $\lambda(X) \leq 1$.

  We also deduce from (4) that
  \begin{equation}
    |\Phi(f)| \leq \Lambda(|f|) = \int_X |f| \, d\lambda = \|f\|_1 \quad (f \in C_c(X)). \tag{5}
  \end{equation}

  The last norm refers to the space $L^1(\lambda)$.
  Thus $\Phi$ is a linear functional on $C_c(X)$ of norm at most 1, with respect to the $L^1(\lambda)$-norm on $C_c(X)$.
  There is a norm-preserving extension of $\Phi$ to a linear functional on $L^1(\lambda)$, and therefore Theorem \ref{thm:Lp_duality} (the case $p = 1$) gives a Borel function $g$, with $|g| \leq 1$, such that
  \begin{equation}
    \Phi(f) = \int_X fg \, d\lambda \quad (f \in C_c(X)). \tag{6}
  \end{equation}

  Each side of (6) is a continuous functional on $C_0(X)$, and $C_c(X)$ is dense in $C_0(X)$.
  Hence (6) holds for all $f \in C_0(X)$, and we obtain the representation (1) with $d\mu = g \, d\lambda$.
\end{proof}

\begin{lemma}[Rudin 6.19]
  \label{lem:norm_eq_variation}
  \lean{ComplexRMK.norm_eq_variation}
  Moreover, the norm of $\Phi$ is the total variation of $\mu$:
  \begin{equation}
    \|\Phi\| = |\mu|(X). \tag{2}
  \end{equation}
\end{lemma}

\begin{proof}
  \uses{def:variation}
  \uses{thm:abs_integral}
  Since $\|\Phi\| = 1$, (6) shows that
  \begin{equation}
    \int_X |g| \, d\lambda \geq \sup \{|\Phi(f)| : f \in C_0(X), \|f\| \leq 1\} = 1. \tag{7}
  \end{equation}
  We also know that $\lambda(X) \leq 1$ and $|g| \leq 1$.
  These facts are compatible only if $\lambda(X) = 1$ and $|g| = 1$ a.e. $[\lambda]$.
  Thus $d|\mu| = |g| \, d\lambda = d\lambda$, by Theorem \ref{thm:abs_integral}, and
  \begin{equation}
    |\mu|(X) = \lambda(X) = 1 = \|\Phi\|, \tag{8}
  \end{equation}
  which proves (2).
\end{proof}
